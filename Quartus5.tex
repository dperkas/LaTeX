\documentclass{article}
\usepackage[utf8]{inputenc}
\usepackage{graphicx}
\usepackage[english,greek]{babel}
\usepackage[colorlinks=true,linkcolor=blue]{hyperref}
\usepackage{listings}
\usepackage{color}

%New colors defined below
\definecolor{codegreen}{rgb}{0,0.6,0}
\definecolor{codegray}{rgb}{0.5,0.5,0.5}
\definecolor{codepurple}{rgb}{0.58,0,0.82}
\definecolor{backcolour}{rgb}{0.95,0.95,0.92}

%Code listing style named "mystyle"
\lstdefinestyle{mystyle}{
  backgroundcolor=\color{backcolour},   commentstyle=\color{codegreen},
  keywordstyle=\color{magenta},
  numberstyle=\tiny\color{codegray},
  stringstyle=\color{codepurple},
  basicstyle=\footnotesize,
  breakatwhitespace=false,         
  breaklines=true,                 
  captionpos=b,                    
  keepspaces=true,                 
  numbers=left,                    
  numbersep=5pt,                  
  showspaces=false,                
  showstringspaces=false,
  showtabs=false,                  
  tabsize=4
}

%"mystyle" code listing set
\lstset{style=mystyle}

\graphicspath{ {C:/Users/Hp/Desktop/quartus/} }
\usepackage{wrapfig}

\title{\foreignlanguage{english}{Quartus Exercise 5 Report}}
\author{Πέρκας Δημήτριος 4156}
\date{\foreignlanguage{english}{May 2019}}

\begin{document}

\maketitle
\thispagestyle{empty}

\clearpage
\thispagestyle{empty}
\tableofcontents
\listoffigures
\clearpage
\pagenumbering{arabic}

\section{Εισαγωγή}

Το κύκλωμα αποτελεί έναν ακολουθιακό πολλαπλαστιαστή. Αποτελείται απο 4 επίπεδα, τον καταχωτητή, αθροιστή, μονάδα ελέγχου και το τελικό κύκλωμα του πολλαπλασιαστή. Η σχεδίαση απαιτεί 5 καταχωρητές, έναν αθροιστή και την μονάδα ελέγχου.

Παρακάτω βρίσκονται όλοι οι κώδικες και όλες τις μονάδες, τα σχηματικά και οι κυμματομορφές, σύμφωνα με τις οδηγίες της εκφώνησης.


\section{Μονάδα Καταχωρητή}

Ο καταχωρητής έχει τις παρακάτω λειτουργίες:
\begin{itemize}
    \item Παράλληλη φόρτωση
    \item Σειριακή φόρτωση
    \item Ολίσθηση
    \item Ασύγχρονη μηδένιση
\end{itemize}


\begin{otherlanguage}{english}
\begin{lstlisting}[language=VHDL, caption= Register VHDL Code]
library IEEE;
use ieee.std_logic_1164.all;

entity Reg is
	generic (n: integer:=4);
	port(
		D_IN: in std_logic_vector (n-1 downto 0);
		SI, CLK, RST, SLOAD, ENABLE: in std_logic;
		SO: out std_logic;
		D_OUT: out std_logic_vector (n-1 downto 0));
end Reg;

architecture RTL of Reg is
signal F: std_logic_vector (n-1 downto 0);
begin
p0: process(RST, CLK)
begin
	if (RST='1') then F<=(n-1 downto 0 => '0');
	elsif (CLK'event and CLK='1') then
		if (ENABLE='1') then
			if (SLOAD='0') then F<=D_IN;
			else F<=SI & F(n-1 downto 1);
			end if;
		end if;
	end if;
end process;
D_OUT<=F;
SO<=F(0);
end RTL;
\end{lstlisting}
\end{otherlanguage}

Δημιουργούμε το σύμβολο, φτιάχνουμε το σχηματικό και τρέχουμε την εξομοίωση:

\begin{figure}[h!]
  \caption{\foreignlanguage{english}{Register Schematic}}
\includegraphics[width = 420pt, height = 180pt]{Reg.PNG} 
\end{figure}

\begin{figure}[h!]
  \caption{\foreignlanguage{english}{Register time waveform}}
\includegraphics[width = 440pt, height = 205pt]{Regwave.PNG}
\end{figure}

\clearpage


\section{Μονάδα αθροιστή}

Στην σχεδίαση του αθροιστή, θα χρειαστεί να βάλουμε και το σήμα \foreignlanguage{english}{CTRL} το οπόιο θα χρειαστεί αργότερα, και καθορίζει αν θα γίνει πρόσθεση.

\begin{otherlanguage}{english}
\begin{lstlisting}[language=VHDL, caption= Adder VHDL Code]
library IEEE;
use ieee.std_logic_1164.all;
use ieee.std_logic_arith.all;
use ieee.std_logic_unsigned.all;

entity Adder is
	generic (n: integer :=4);
	port
	(
		CTRL	: in std_logic;
		A, B	: in std_logic_vector(n-1 downto 0);
		F		: out std_logic_vector(n-1 downto 0);
		COUT	: out std_logic
	);
end Adder;

architecture RTL of Adder is
	signal Interm : std_logic_vector(n downto 0);
begin

p0: process(A, B, CTRL)
begin
	if (CTRL= '0') then Interm<= '0' & B;
	else Interm<=('0' & A)+('0' & B);
	end if;
end process;
F<=Interm(n-1 downto 0);
COUT<=Interm(n);
end RTL;
\end{lstlisting}
\end{otherlanguage}



\begin{figure}[h!]
  \caption{\foreignlanguage{english}{Adder Schematic}}
\includegraphics[width = 420pt, height = 180pt]{adder.PNG} 
\end{figure}


\clearpage

\section{Μονάδα Ελέγχου}

Η συγκεκριμένη μονάδα είναι υπεύθυνη για τον συντονισμό των άλλων και καθορίζει πότε θα γίνει η κάθε λειτουργία.

\begin{otherlanguage}{english}
\begin{lstlisting}[language=VHDL, caption= Control Unit VHDL Code]
library IEEE;
use ieee.std_logic_1164.all;
use ieee.std_logic_arith.all;
use ieee.std_logic_unsigned.all;


entity CtrlLogic is
	generic ( n: integer := 4 );
	port (
	Rst, CLK : in std_logic;
	SL_A, SL_B, SL_H, SL_L, SL_C: out std_logic;
	EN_A, EN_B, EN_H, EN_L, EN_C: out std_logic );
end CtrlLogic;

architecture RTL of CtrlLogic is
	type state_type is (LOAD, MULT, HOLD);
	type mult_type is (SHIFT, ADD);
	signal state : state_type;
	signal m_state : mult_type;
	signal count : std_logic_vector (n-1 downto 0);

begin
	
p0: process(Rst, CLK)
begin
	if (Rst='1') then
	state<=LOAD;
	count<=(n-1 downto 0 => '0');
	elsif (CLK'event and CLK='1') then
		case state is
			when LOAD => if (conv_integer(count)<2*n-1) then count<=count+1;
				else count<=(n-1 downto 0 => '0');
					state<=MULT; m_state<=ADD;
				end if;
			when MULT => if (m_state=ADD) then m_state<=SHIFT;
				elsif (m_state=SHIFT) then
					m_state<=ADD;
					if (conv_integer(count)<n-1) then count<=count+1;
					else state<=HOLD;
					end if;
				end if;
			when HOLD => null;
		end case;
	end if;
end process;

EN_A<='1' when state=LOAD else '0';
SL_A<='1'when state=LOAD else '0';

EN_B<='1'	when (state=LOAD	or (state=MULT	and m_state=SHIFT))	else '0';
SL_B<='1'	when (state=LOAD	or (state=MULT	and m_state=SHIFT))	else '0';	

EN_H<='1'	when state=MULT	else '0';
SL_H<='1'	when m_state=SHIFT	else '0';	

EN_L<='1'	when (state=MULT	and m_state=SHIFT)	else '0';
SL_L<='1'	when (state=MULT	and m_state=SHIFT)	else '0';

EN_C<='1'	when (state=MULT	and m_state=ADD)	else '0';
SL_C<='0';

end RTL;
\end{lstlisting}
\end{otherlanguage}

Παρακάτω βλέπουμε το σχηματικό, την κυματομορφή και τις καθυστερήσεις:

\begin{figure}[h!]
  \caption{\foreignlanguage{english}{Control Unit schematic}}
\includegraphics[width = 420pt, height = 180pt]{crtl.PNG} 
\end{figure}

\clearpage

\section{Σχεδίαση Πολλαπλαστιαστή}
Στην τελική σχεδίαση του πολλαπλασιαστή, πρέπει να βάλουμε σαν \foreignlanguage{english}{components} όλες τις παραπάνω μονάδες για να τις διασυνδέσουμε και να πάρουμε το τελικό κύκλωμα.


\begin{otherlanguage}{english}
\begin{lstlisting}[language=VHDL, caption= Multiplier Code VHDL]
library IEEE;
use ieee.std_logic_1164.all;
use ieee.std_logic_arith.all;
use ieee.std_logic_unsigned.all;

entity Multiplier is
	generic (n: integer :=4);
	port (
		Rst, CLK, SI : in std_logic;
		Low, High : out std_logic_vector (n-1 downto 0));
end Multiplier;

architecture RTL of Multiplier is

component CtrlLogic
	generic ( n: integer := 4 );
	port (
		Rst, CLK : in std_logic;
		SL_A, SL_B, SL_H, SL_L, SL_C : out std_logic;
		EN_A, EN_B, EN_H, EN_L, EN_C : out std_logic );
end component;

component Adder
	generic (n: integer :=4);
	port (
		CTRL : in std_logic;
		A, B : in std_logic_vector (n-1 downto 0);
		F : out std_logic_vector (n-1 downto 0);
		COUT : out std_logic );
end component;

component Reg
	generic (n: integer:=4);
	port (
		D_IN: in std_logic_vector (n-1 downto 0);
		SI, CLK, RST, SLOAD, ENABLE: in std_logic;
		SO: out std_logic;
		D_OUT: out std_logic_vector (n-1 downto 0));
end component;

signal SL_A, EN_A, SO_A, SL_B, EN_B, SO_B, SL_C, EN_C, SO_C: std_logic;
signal SL_H, EN_H, SO_H, SL_L, EN_L, SO_L: std_logic;
signal A,B,F_ADD, H: std_logic_vector (n-1 downto 0);
signal C, COUT : std_logic_vector (0 downto 0);

begin
uA: Reg generic map(n=>4)
	port map( D_IN=>(n-1 downto 0=>'0'), SI=>SI, CLK=>CLK, RST=>RST,SLOAD=>SL_A,
			ENABLE=>EN_A, SO=>SO_A, D_OUT=>A);
uB: Reg generic map(n=>4)
	port map(D_IN=>(n-1 downto 0=>'0'), SI=>SO_A, CLK=>CLK, RST=>RST,SLOAD=>SL_B,
			ENABLE=>EN_B, SO=>SO_B, D_OUT=>B);
uAdder: Adder generic map(n=>4)
	port map(CTRL=>B(0), A=>A, B=>H, F=>F_ADD, COUT=>COUT(0));
uC: Reg generic map(n=>1)
	port map(D_IN=>COUT, SI=>'0', CLK=>CLK, RST=>RST, SLOAD=>SL_C, ENABLE=>EN_C,
			SO=>SO_C, D_OUT=>C);
uH: Reg generic map(n=>4)
	port map(D_IN=>F_ADD, SI=>SO_C, CLK=>CLK, RST=>RST, SLOAD=>SL_H, ENABLE=>EN_H,
			SO=>SO_H,D_OUT=>H);
uL: Reg generic map(n=>4)
	port map(D_IN=>(n-1 downto 0=>'0'), SI=>S0_H, CLK=>CLK, RST=>RST, SLOAD=>SL_L,
			ENABLE=>EN_L, SO=>SO_L, D_OUT=>low);
uCTRL: CtrlLogic generic map(n=>4)
	port map (Rst=>RST, CLK=>CLK, SL_A=>SL_A, SL_B=>SL_B, SL_H=>SL_H, SL_L=>SL_L,
			SL_C=>SL_C, EN_A=>EN_A, EN_B=>EN_B, EN_H=>EN_H, EN_L=>EN_L, EN_C=>EN_C);
			
High<=H;

end RTL;
\end{lstlisting}
\end{otherlanguage}


\clearpage


\section{Παρατηρήσεις}

$\bullet$ Κατα την διάρκεια του \foreignlanguage{english}{project} οι κυματομορφές του αθροιστή, της μονάδας ελέγχου και του πολλαπλασιαστή λειτουργούσαν κανονικά και έβγαζαν τα αποτελέσματα που αναγράφονται και στην εκφώνιση, αφου όλα τα βήματα ακολουθήθηκαν ρητά.

Ωστόσο, έχοντας τελείωσει την εργασία, όταν ξεκίνησα το \foreignlanguage{english}{report} και πήγα να τρέξω τις κυματομορφές και τις καθυστερήσεις,ο \foreignlanguage{english}{compiler} μου έβγαζε προβλήματα, γιατι είχα ανεβεί στάδια στην ιεραρχία και λόγω κάποιου προβλήματος στο τελευταίο επίπεδο, δεν έτρεχαν και τα προηγούμενα.
\end{document}

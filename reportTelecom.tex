\documentclass{article}
\usepackage[utf8]{inputenc}
\usepackage{graphicx}
\usepackage[english,greek]{babel}
\usepackage[colorlinks=true,linkcolor=blue]{hyperref}
\usepackage{listings}
\usepackage{color}
\usepackage{verbatim}
\newenvironment{metaverbatim}{\verbatim}{\endverbatim}

\definecolor{codegreen}{rgb}{0,0.6,0}
\definecolor{codegray}{rgb}{0.5,0.5,0.5}
\definecolor{codepurple}{rgb}{0.58,0,0.82}
\definecolor{backcolour}{rgb}{0.95,0.95,0.92}

%Code listing style named "mystyle"
\lstdefinestyle{mystyle}{
  backgroundcolor=\color{backcolour},   commentstyle=\color{codegreen},
  keywordstyle=\color{magenta},
  numberstyle=\tiny\color{codegray},
  stringstyle=\color{codepurple},
  basicstyle=\footnotesize,
  breakatwhitespace=false,         
  breaklines=true,                 
  captionpos=b,                    
  keepspaces=true,                 
  numbers=left,                    
  numbersep=5pt,                  
  showspaces=false,                
  showstringspaces=false,
  showtabs=false,                  
  tabsize=4
}

%"mystyle" code listing set
\lstset{style=mystyle}

\graphicspath{ {C:/Users/Hp/Desktop/quartus/} }
\usepackage{wrapfig}



%\usepackage[left=2cm,right=2cm,top=2cm,bottom=2cm]{geometry} IT ALL GOES LEFT





  
\begin{document}




\title{ΤΗΛΕΠΙΚΟΙΝΩΝΙΑΚΑ ΣΥΣΤΗΜΑΤΑ \\ Δεύτερη σειρά ασκήσεων}
\author{Πέρκας Δημήτριος}
\date{Μάιος 2020}

\maketitle

\pagebreak


\tableofcontents
\newpage

%\part{Ασκήσεις}




\section{Κώδικας \selectlanguage{english}{Matlab}}

\begin{otherlanguage}{english}
\begin{lstlisting}[language=MATLAB, caption= BER In BPAM Modulation ]
%PERKAS DIMITRIOS 4156
clear all 
close all

N=10^7;
err=0;
Eb=1;
EbNo_dB=8;
EbNo=10^(EbNo_dB/10);
s1=-sqrt(Eb);
s2=sqrt(Eb);


for experiment = 1:N

% Symbol Generation for BPAM
y=randi(2)-1;
if y==0
    s=s1;
elseif y==1 
    s=s2;
end

% Additive White Gaussian Noise     
sigma=sqrt(Eb/(2*EbNo));
n=sigma*randn;
r=s+n;

% Symbol Detection for BPAM
if r<0
    s_cup=s1;
elseif r>0
    s_cup=s2;
end


if s_cup~=s
    err=err+1;
end

end

% BER Calculation
simulatedBER=err/N
Pb = qfunc(sqrt(2*EbNo))
\end{lstlisting}
\end{otherlanguage}




\selectlanguage{greek}{
Ο αλγόριθμος πραγματοποιήθηκε $N=10^7$ φορές. Η αλλαγή του \foreignlanguage{english}{EbNo\_dB} γίνεται κάθε
φορά χειροκίνητα για να πάρουμε το κάθε αποτέλεσμα.

Η τιμη \foreignlanguage{english}{simulatedBER} αφορά το πειραματικό \foreignlanguage{english}{BER}, το οποίο είναι ο αριθμός των εσφαλμένων \foreignlanguage{english}{bits}(ο μετρητής), πρός τον αριθμό $Ν$ των επαναλήψεων. \\
Το θεωρητικό \foreignlanguage{english}{BER} δίνεται με την χρήση της συνάρτησης $Q$ απο τον τύπο:
}

\[P_b=Q\left(\sqrt{\frac{2E_b}{N_0}}\right)\] \\ \\ 

\section{Αποτελέσματα}

Θέτουμε στην τιμή του \foreignlanguage{english}{EbNo\_dB} το 5 και παίρνουμε:
\foreignlanguage{english}{
\begin{verbatim}
simulatedBER = \\
    0.0060\\
Pb =\\
    0.0060\\
\end{verbatim}}

Άλλοτε η τιμή του πειραματικού \foreignlanguage{english}{BER} συμπίπτει ακριβώς με του θεωτητικού, άλλοτε παίρνουμε αποτελέσματα με απόκληση $0.0001-0.0002$ \\ \\
Για \foreignlanguage{english}{EbNo\_dB}$=6$:
\foreignlanguage{english}{
\begin{verbatim}
simulatedBER = \\
    0.0024\\
Pb =\\
    0.0024\\
\end{verbatim}}

Για \foreignlanguage{english}{EbNo\_dB}$=7$:
\foreignlanguage{english}{
\begin{verbatim}
simulatedBER = \\
    7.5720e-04\\
Pb =\\
    7.7267e-04\\
\end{verbatim}}

Για \foreignlanguage{english}{EbNo\_dB}$=8$:
\foreignlanguage{english}{
\begin{verbatim}
simulatedBER = \\
    1.9920e-04\\
Pb =\\
    1.9091e-04\\
\end{verbatim}}

\end{document}

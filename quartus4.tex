\documentclass{article}
\usepackage[utf8]{inputenc}
\usepackage{graphicx}
\usepackage[english,greek]{babel}
\usepackage[colorlinks=true,linkcolor=blue]{hyperref}
\usepackage{listings}
\usepackage{color}

%New colors defined below
\definecolor{codegreen}{rgb}{0,0.6,0}
\definecolor{codegray}{rgb}{0.5,0.5,0.5}
\definecolor{codepurple}{rgb}{0.58,0,0.82}
\definecolor{backcolour}{rgb}{0.95,0.95,0.92}

%Code listing style named "mystyle"
\lstdefinestyle{mystyle}{
  backgroundcolor=\color{backcolour},   commentstyle=\color{codegreen},
  keywordstyle=\color{magenta},
  numberstyle=\tiny\color{codegray},
  stringstyle=\color{codepurple},
  basicstyle=\footnotesize,
  breakatwhitespace=false,         
  breaklines=true,                 
  captionpos=b,                    
  keepspaces=true,                 
  numbers=left,                    
  numbersep=5pt,                  
  showspaces=false,                
  showstringspaces=false,
  showtabs=false,                  
  tabsize=4
}

%"mystyle" code listing set
\lstset{style=mystyle}

\graphicspath{ {C:/Users/Hp/Desktop/quartus/} }
\usepackage{wrapfig}

\title{\foreignlanguage{english}{Quartus Exercise 4 Report}}
\author{Πέρκας Δημήτριος}
\date{\foreignlanguage{english}{April 2019}}

\begin{document}

\maketitle
\thispagestyle{empty}

\clearpage
\thispagestyle{empty}
\tableofcontents
\listoffigures
\clearpage
\pagenumbering{arabic}

\section{Μέρος 1ο: Συνδιαστικά Κυκλώματα}

\subsection{Ερώτημα 1ο, πολυπλέκτης 4-σε-1}

Αφού γράψουμε τον κώδικα για τον πολυπλέκτη, δημιουργούμε το αντίστοιχο \foreignlanguage{english}{block} για αυτόν και στη συνέχεια φτιάχνουμε το παρακάτω σχηματικό:

\begin{figure}[h!]
  \caption{\foreignlanguage{english}{Mux 4-to-1 schematic}}
\includegraphics[width = 420pt, height = 180pt]{mux4-1Schem.PNG} 
\end{figure}

Με την χρονική εξομοίωση βλέπουμε πως το κύκλωμα λειτουργεί σωστά
\begin{figure}[h!]
  \caption{\foreignlanguage{english}{Mux 4-to-1 time waveform}}
\includegraphics[width = 440pt, height = 205pt]{mux4-1WAVE.PNG}
\end{figure}

\subsection{Ερώτημα 2ο, πολυπλέκτης 16-σε-1}

Αφου έχουμε κανει τον πολυπλέκτη 4-σε-1, με την χρήση \foreignlanguage{english}{structural VHDL} φτιάχνουμε τον πολυπλέκτη 16-1. Αφού περάσουμε τον κώδικα, δημιουργούμε το σχηματικό:

\begin{figure}[h!]
  \caption{\foreignlanguage{english}{Mux 16-to-1 schematic}}
\includegraphics[width = 420pt, height = 180pt]{mux16-1Schem.PNG} 
\end{figure}

Η παρακάτω εξομοίωση έγινε στο πεδίο της συχνότητας έτσι ώστε να μπορεί να ελεχθεί ευκολότερα η λειτουργία του πολυπλέκτη:

\begin{figure}[h!]
  \caption{\foreignlanguage{english}{Mux 16-to-1 frequency waveform}}
\includegraphics[width = 430pt, height = 200pt]{mux16_1Wave.PNG}
\end{figure}

\clearpage

Για τον έλεγχο των επιδώσεων του πολυπλέκτη, ανατρέχουμε στο \foreignlanguage{english}{Classic Timing Analyzer Tool} και παίρνουμε τις εξής καθυστερήσεις:

\begin{figure}[h!]
  \caption{\foreignlanguage{english}{Mux 16-to-1 Delays}}
\includegraphics[width = 170pt, height = 220pt]{mux16_1DELAYS.PNG}
\end{figure}

Βλέπουμε πως η μέγιστη καθυστέρηση είναι απο την επιλογή \foreignlanguage{english}{SEL[1]} στην έξοδο \foreignlanguage{english}{F}, η οποία είναι $15.346ns$

\clearpage

\subsection{Ερώτημα 3ο, Αθροιστής}

Αρχικά δημιουργούμε έναν πλήρη αθροιστή σε \foreignlanguage{english}{VHDL}:

\begin{otherlanguage}{english}
\begin{lstlisting}[language=VHDL, caption= Full Adder]
library ieee;
use ieee.std_logic_1164.all;
use IEEE.std_logic_unsigned.all;

entity full_adder is port(
	A: in std_logic_vector(1 downto 0);
	B: in std_logic_vector(1 downto 0);
	cin: in std_logic;
	sum: out std_logic_vector(1 downto 0);
	cout: out std_logic);
end full_adder;

architecture RTL of full_adder is	
signal InternalSum: std_logic_vector(2 downto 0);
begin

	InternalSum<="0"&A + "0"&B +"00"&cin;
	Sum<=InternalSum(1 downto 0);
	Cout<=InternalSum(2);
	
end RTL;
\end{lstlisting}
\end{otherlanguage}

Η εξομοίωση επιβεβαιώνει την λειτουργία του αθροιστή:

\begin{figure}[h!]
  \caption{\foreignlanguage{english}{FA waveform}}
\includegraphics[width = 440pt, height = 200pt]{faWaveF.PNG}
\end{figure}

\clearpage

Οι καθυστερήσεις του πλήρη αθροιστή φαίνονται παρακάτω:

\begin{figure}[h!]
  \caption{\foreignlanguage{english}{FA Delays}}
\includegraphics[width = 180pt, height =90pt]{FAtime.PNG}
\end{figure}


Αφού ολοκρηρώσουμε τον πλήρη αθροιστή, θα χρησιμοποιήσουμε \foreignlanguage{english}{structural VHDL} για να δημιουργήσουμε έναν αθροιστή των \foreignlanguage{english}{8 bits}. Ο αρθοιστής αυτός θα χρησιμοποιήσει 8 πλήρεις αθροιστές. Ο πλήρης αθροιστής θα δηλωθεί σαν \foreignlanguage{english}{component} στο σώμα αρχιτεκτονικής του 8-μπιτου αθροιστή.

\begin{otherlanguage}{english}
\begin{lstlisting}[language=VHDL, caption= 8-bit Adder With Structural VHDL]
library IEEE;
use IEEE.std_logic_1164.all;
use IEEE.std_logic_unsigned.all;

entity Adder8Bit is
	port(
		a, b: in std_logic_vector(7 downto 0);
		cin: in std_logic;
		sum: out std_logic_vector(7 downto 0);
		cout: out std_logic);
end Adder8Bit;

architecture RTL of Adder8Bit is
component full_adder
	port(
		a :in std_logic_vector(1 downto 0);
		b :in std_logic_vector(1 downto 0);
		cin:in std_logic;
		sum:out std_logic_vector(1 downto 0);
		cout: out std_logic);
end component;

signal c_int: std_logic_vector(7 downto 1);

begin
	FA0: full_adder port map (a(0),b(0),cin,sum(0),c_int(1));
	FA1: full_adder port map (a(1),b(1),c_int(1),sum(1),c_int(2));
	FA2: full_adder port map (a(2),b(2),c_int(2),sum(2),c_int(3));
	FA3: full_adder port map (a(3),b(3),c_int(3),sum(3),c_int(4));
	FA4: full_adder port map (a(4),b(4),c_int(4),sum(4),c_int(5));
	FA5: full_adder port map (a(5),b(5),c_int(5),sum(5),c_int(6));
	FA6: full_adder port map (a(6),b(6),c_int(6),sum(6),c_int(7));
	FA7: full_adder port map (a(7),b(7),c_int(7),sum(7),cout);
end RTL;
\end{lstlisting}
\end{otherlanguage}

\clearpage

Παρακάτω βλέπουμε το σχηματικό και την χρονική εξομοίωση:

\begin{figure}[h!]
  \caption{\foreignlanguage{english}{8-bit Adder schematic}}
\includegraphics[width = 420pt, height = 180pt]{adder8bitsSchem.PNG} 
\end{figure}

\begin{figure}[h!]
  \caption{\foreignlanguage{english}{8-bit Adder waveform}}
\includegraphics[width = 490pt, height = 230pt]{adder8bitsWave.PNG}
\end{figure}


\subsection{Ερώτημα 4ο, Διαφορετική υλοποίηση αθροιστή}

Εκτός απο την \foreignlanguage{english}{structural} υλοποίηση του αθροιστή με λογική, παρακάτω βρίσκεται ο κώδικας του 8-μπιτου αθροιστή με τον τελεστή "$+$":

\begin{otherlanguage}{english}
\begin{lstlisting}[language=VHDL, caption= 8-bit Adder]
Library IEEE;
use IEEE.std_logic_1164.all;
use IEEE.std_logic_unsigned.all;

entity Adder_8bit1 is
	port(
		A: in std_logic_vector(7 downto 0);
		B: in std_logic_vector(7 downto 0);
		cin: in std_logic;
		Sum: out std_logic_vector(7 downto 0);
		cout: out std_logic);
end Adder_8bit1;

architecture RTL of Adder_8bit1 is
	signal InternalSum: std_logic_vector(8 downto 0);
begin
	InternalSum<="0"&A + "0"&B +"00000000"&cin;
	Sum<=InternalSum(7 downto 0);
	cout<=InternalSum(8);
end RTL;
\end{lstlisting}
\end{otherlanguage}

\clearpage

\section{Μέρος 2ο}


\subsection{Ερώτημα 1ο, \foreignlanguage{english}{D Flip Flop - D Latch}}

Αφού γράψουμε τον κώδικα του \foreignlanguage{english}{D Flip Flop}, δημιουργούμε το σύμβολο, βάζουμε σε νέο σχηματικό τα \foreignlanguage{english}{inputs, outputs} και ελέγχουμε την λειτουργία του με την εξομοίωση:

\begin{figure}[h!]
  \caption{\foreignlanguage{english}{D Flip Flop schematic}}
\includegraphics[width = 420pt, height = 180pt]{DffSchem.PNG} 
\end{figure}

\begin{figure}[h!]
  \caption{\foreignlanguage{english}{D Flip Flop waveform}}
\includegraphics[width = 430pt, height = 210pt]{DffWave.PNG}
\end{figure}

Το κύκλωμα λειτουργεί σωστά. \\ \\ 

Όμοια δημιουργούμε και το \foreignlanguage{english}{D Latch}, ελέγχουμε την λειτουργία του, και στη συνέχεια τα ενώνουμε στο σχηματικό με τον παρακάτω τρόπο:

\begin{figure}[h!]
  \caption{\foreignlanguage{english}{D Flip Flop - Latch schematic}}
\includegraphics[width = 420pt, height = 180pt]{DffLatchSchem.PNG} 
\end{figure}

Κάνουμε την ανάλογη εξομοίωση και παρατηρούμε πως το κύκλωμα λειτουργεί σωστά, και σημειώνουμε τις καθυστερήσεις με \foreignlanguage{english}{time bars}:

\begin{figure}[h!]
  \caption{\foreignlanguage{english}{D Flip Flop - Latch waveform}}
\includegraphics[width = 420pt, height = 200pt]{DffLatchWave.PNG}
\end{figure}

\clearpage

\subsection{Ερώτημα 2ο, Κύκλωμα Καταχωρητή}

Ο καταχωρητής σχεδιάστηκε με \foreignlanguage{english}{behavioral VHDL}, παρακάτω βρίσκεται ο κώδικας:

\begin{otherlanguage}{english}
\begin{lstlisting}[language=VHDL, caption= Register with begavioral VHDL]
library IEEE;
use IEEE.std_logic_1164.all;

entity Reg8 is port(
	D : in std_logic_vector(7 downto 0);
	CLK, CLR, SET: in std_logic;
	Q, Qn : out std_logic_vector(7 downto 0));
end Reg8;

architecture behavioral of Reg8 is
	signal DFF : std_logic_vector(7 downto 0);
	
begin
	seq0 : process(CLK, CLR, SET )
	begin
		if (CLR='1') then DFF<="00000000";
		elsif (SET='1') then DFF<="11111111";
		elsif (CLK'event and CLK = '1') then DFF <=D;
		end if;
	end process;
	Q <= DFF; Qn <= not DFF;
end behavioral;
\end{lstlisting}
\end{otherlanguage}

Δημιουργούμε το σύμβολο, τρέχουμε τις εξομοιώσεις και παρατηρούμε την σωστή λειτουργία του καταχωρητή:

\begin{figure}[h!]
  \caption{\foreignlanguage{english}{Register schematic}}
\includegraphics[width = 420pt, height = 180pt]{reg8schem.PNG} 
\end{figure}


\begin{figure}[h!]
  \caption{\foreignlanguage{english}{Register waveform}}
\includegraphics[width = 440pt, height = 230pt]{Reg8Wave.PNG}
\end{figure}

\clearpage


Με το \foreignlanguage{english}{Classic Timing Analyzer Tool} παίρνουμε τις εξής σύχγρονες και ασύγχρονες καθυστερήσεις:


\begin{figure}[h!]
  \caption{\foreignlanguage{english}{Register Delays}}
\includegraphics[width = 270pt, height = 130pt]{reg8delays.PNG} 
\end{figure}


\begin{figure}[h!]
  \caption{\foreignlanguage{english}{Register Asynchronous Delays}}
\includegraphics[width = 270pt, height = 200pt]{reg8delaysAsynch.PNG} 
\end{figure}

\begin{figure}[h!]
  \caption{\foreignlanguage{english}{Register Synchronous Delays}}
\includegraphics[width = 270pt, height = 130pt]{reg8delaysSyn.PNG} 
\end{figure}

\clearpage

\subsection{Ερώτημα 3ο, Κύκλωμα Μετρητή}

Θα σχεδιάσουμε έναν 8-μπιτο μετρητή ο οποίος θα μετράει από $0-255$. Έχει όλες τις ζητούμενες δυνατότητες.

Για τον σχεδιασμό του, χρησιμοποιήθηκαν 2 \foreignlanguage{english}{processes}, μία για την μέτρηση και μια για τις λειτουργίες ασύγχρονης θέσης και μηδένισης.

\begin{otherlanguage}{english}
\begin{lstlisting}[language=VHDL, caption= VHDL Counter ]
library IEEE;
use IEEE.std_logic_1164.all;
use IEEE.std_logic_unsigned.all;
use IEEE.std_logic_arith.all;

entity Counter8 is port(
	CLK: in std_logic;
	CLR: in std_logic;
	SET: in std_logic;
	count: out std_logic_vector(7 downto 0));
end Counter8;

architecture RTL of Counter8 is
begin

counter: process
	variable value: std_logic_vector(7 downto 0):="00000000";
	begin
	
	loop
		wait until CLK='1';
		value:=value+1;
		count<=value;
	end loop;
	end process;
	

asynch: process(CLK, CLR, SET)
	begin
		if (CLR='1') then count<="00000000";
		elsif (SET='1') then count<="11111111";
		end if;
	end process;

end RTL;
\end{lstlisting}
\end{otherlanguage}


\clearpage

Δημιουργήθηκε το παρακάτω σχηματικό μετα την δημιουργία του \foreignlanguage{english}{block}:

\begin{figure}[h!]
  \caption{\foreignlanguage{english}{Counter schematic}}
\includegraphics[width = 420pt, height = 180pt]{CounterSchem.PNG} 
\end{figure}







\clearpage


\section{Παρατηρήσεις}

$\bullet$ Μετά την δημιουργία του \foreignlanguage{english}{D Flip Flop}, αφού έχουμε πάρει τις καθυστερήσεις, στο \foreignlanguage{english}{Classic Timing Analyzer Tool} δεν εμφανίζεται κάποια απόδοση του \foreignlanguage{english}{flip flop}, και ο δείκτης είναι σταθερός στο $0$

\begin{figure}[h!]
  \caption{\foreignlanguage{english}{D Flip Flop Performance}}
\includegraphics[width = 470pt, height = 240pt]{DffPerformance.PNG}
\end{figure} \\ \\ 

$\bullet$ Κατά την δημιουργία των κυματομορφών του μετρητή παρουσιάστηκαν κάποια προβλήματα με τον \foreignlanguage{english}{compiler}, για αυτό δεν προσθέθηκε κυματομορφή στο \foreignlanguage{english}{report}. Ωστόσο, ο κώδικας αρχικά δεν έχει προβλήματα, αφου δημιουργήθηκε και το ανάλογο \foreignlanguage{english}{block}.

\end{document}

%-----code environment lstlisting with VHDL example
%-----need \usepackage{listings}
%-----need this code for colored code environment:
%-----need \usepackage{color} for colored code

%----------------------------------O------------------------------------
%New colors defined below
\definecolor{codegreen}{rgb}{0,0.6,0}
\definecolor{codegray}{rgb}{0.5,0.5,0.5}
\definecolor{codepurple}{rgb}{0.58,0,0.82}
\definecolor{backcolour}{rgb}{0.95,0.95,0.92}

%Code listing style named "mystyle"
\lstdefinestyle{mystyle}{
  backgroundcolor=\color{backcolour},   commentstyle=\color{codegreen},
  keywordstyle=\color{magenta},
  numberstyle=\tiny\color{codegray},
  stringstyle=\color{codepurple},
  basicstyle=\footnotesize,
  breakatwhitespace=false,         
  breaklines=true,                 
  captionpos=b,                    
  keepspaces=true,                 
  numbers=left,                    
  numbersep=5pt,                  
  showspaces=false,                
  showstringspaces=false,
  showtabs=false,                  
  tabsize=4
}

%"mystyle" code listing set
\lstset{style=mystyle}
%----------------------------------O------------------------------------
 
 
 
 
\begin{lstlisting}[language=VHDL, caption= Register VHDL Code]
library IEEE;
use ieee.std_logic_1164.all;

entity Reg is
	generic (n: integer:=4);
	port(
		D_IN: in std_logic_vector (n-1 downto 0);
		SI, CLK, RST, SLOAD, ENABLE: in std_logic;
		SO: out std_logic;
		D_OUT: out std_logic_vector (n-1 downto 0));
end Reg;

architecture RTL of Reg is
signal F: std_logic_vector (n-1 downto 0);
begin
p0: process(RST, CLK)
begin
	if (RST='1') then F<=(n-1 downto 0 => '0');
	elsif (CLK'event and CLK='1') then
		if (ENABLE='1') then
			if (SLOAD='0') then F<=D_IN;
			else F<=SI & F(n-1 downto 1);
			end if;
		end if;
	end if;
end process;
D_OUT<=F;
SO<=F(0);
end RTL;
\end{lstlisting}

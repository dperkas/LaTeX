\documentclass{article}
\usepackage[utf8]{inputenc}
\usepackage[english,greek]{babel}
\usepackage[colorlinks=true,linkcolor=blue]{hyperref}
\usepackage{listings}
\usepackage{color}
\usepackage{verbatim}
\newenvironment{metaverbatim}{\verbatim}{\endverbatim}

\definecolor{codegreen}{rgb}{0,0.6,0}
\definecolor{codegray}{rgb}{0.5,0.5,0.5}
\definecolor{codepurple}{rgb}{0.58,0,0.82}
\definecolor{backcolour}{rgb}{0.95,0.95,0.92}

%Code listing style named "mystyle"
\lstdefinestyle{mystyle}{
  backgroundcolor=\color{backcolour},   commentstyle=\color{codegreen},
  keywordstyle=\color{magenta},
  numberstyle=\tiny\color{codegray},
  stringstyle=\color{codepurple},
  basicstyle=\footnotesize,
  breakatwhitespace=false,         
  breaklines=true,                 
  captionpos=b,                    
  keepspaces=true,                 
  numbers=left,                    
  numbersep=5pt,                  
  showspaces=false,                
  showstringspaces=false,
  showtabs=false,                  
  tabsize=4
}

%"mystyle" code listing set
\lstset{style=mystyle}
\usepackage{graphicx}
\graphicspath{C:\Users\Hp\Desktop\matlabPics} 

\usepackage{wrapfig}



%\usepackage[left=2cm,right=2cm,top=2cm,bottom=2cm]{geometry} IT ALL GOES LEFT





  
\begin{document}




\title{ΤΗΛΕΠΙΚΟΙΝΩΝΙΑΚΑ ΣΥΣΤΗΜΑΤΑ \\Εργαστηριακή Άσκηση}
\author{Πέρκας Δημήτριος 4156}
\date{Μάιος 2020}

\maketitle

\pagebreak


\tableofcontents
\newpage


\section{Μέρος Πρώτο}

\subsection{Κώδικας \selectlanguage{english}{Matlab}}
\begin{otherlanguage}{english}
\begin{lstlisting}[language=MATLAB, caption= \foreignlanguage{greek}{Μέρος Πρώτο}  ]
%PERKAS DIMITRIOS 4156

clear all
close all

%% i)
fs=10000;
Ts=1/fs;
t=[0:Ts:10];
m=cos(2*pi*t);

plot(t,m)

%% ii)
Ac=2;
fc=500;
Ts=1/fs;
t=[0:Ts:10];
x=(Ac+cos(2*pi*t)).*cos(2*pi*fc*t);
plot(t,x)

%% iii)
env=abs(hilbert(x));
t=[0:Ts:10];
plot(t,env)


%% iv)
t=[0:Ts:10];
plot(t,env-2)


%% II(ii)
Ac=0.7;
fc=500;
Ts=1/fs;
t=[0:Ts:10];
x=(Ac+cos(2*pi*t)).*cos(2*pi*fc*t);
plot(t,x)


%% II(iii)
env=abs(hilbert(x));
plot(t,env)


\end{lstlisting}
\end{otherlanguage}

\pagebreak

\subsection{Αποτελέσματα}
Παρακάτω βλέπουμε τις γραφικές παραστάσεις που παίρνουμε στην έξοδο:\\
α)\\
\includegraphics[width=14cm, height=8cm]{1.png}
β)\\
\includegraphics[width=14cm, height=8cm]{2.png}
γ)\\
\includegraphics[width=14cm, height=8cm]{3.png}
δ)Παρατηρούμε οτι δεν παίρνουμε το αρχικό μας σήμα πίσω. Δεν έχουμε υπερδιαμόρφωση, διότι
\[\mu=\frac{|\max\{m(t)\}|}{A_c}=\frac{1}{2}<1 \]
Αρα μπορούμε να πάρουμε το αρχικό μας σήμα πίσω, αφαιρώντας το $A_c=2$, κάνοντας δλαδή \foreignlanguage{english}{env-2}
Παίρνουμε, λοιπόν, το αρχικό μας σήμα.\\
\includegraphics[width=14cm, height=8cm]{4.png}
\pagebreak

ε-β)\\
Για $A_c=0.7$:\\
\includegraphics[width=14cm, height=8cm]{2b.png}
ε-γ)\\
\includegraphics[width=14cm, height=8cm]{2c.png}
ε-δ)\\
Βλέπουμε οτι ούτε σε αυτή τη περίπτωση δεν παίρνουμε το αρχικό μας σήμα πίσω. Αυτή τη φορά όμως έχουμε υπερδιαμόρφωση, διότι:
\[\mu=\frac{|\max\{m(t)\}|}{A_c}=\frac{1}{0.7}=1.42>1 \]
Δεν μπορούμε, συνεπώς, με $A_c=0.7$ να πάρουμε το αρχικό μας σήμα πίσω.
















\section{Μέρος Δεύτερο}

\subsection{Κώδικας \selectlanguage{english}{Matlab}}

\begin{otherlanguage}{english}
\begin{lstlisting}[language=MATLAB, caption= SER In QPSK Modulation ]
%PERKAS DIMITRIOS 4156
clear all
close all

N=10^6;
Es=1;
EsNo_dB=10;
EsNo=10^(EsNo_dB/10);
err=0;

s1=(sqrt(2*Es))/2 + j*(sqrt(2*Es))/2;
s2=(sqrt(2*Es))/2 - j*(sqrt(2*Es))/2;
s3=-(sqrt(2*Es))/2 - j*(sqrt(2*Es))/2;
s4=-(sqrt(2*Es))/2 + j*(sqrt(2*Es))/2;

for experiment = 1:N
    %Symbol Generation For QPSK
    y=randi(4);
    if y==1
        s=s1;
    elseif y==2
        s=s2;
    elseif y==3
        s=s3;
    elseif y==4
        s=s4;
    end
    
    % Additive White Gaussian Noise     
    sigma=sqrt(Es/(2*EsNo));
    n1=sigma*randn;
    n2=sigma*randn;
    n=n1+j*n2;
    r=s+n;
    
    % Symbol Detection for QPSK
    if (real(r)>0 && imag(r)>0)
        s_cup=s1;
    elseif(real(r)<0 && imag(r)>0)
        s_cup=s4;
    elseif(real(r)>0 && imag(r)<0)
        s_cup=s2;
    elseif(real(r)<0 && imag(r)<0)
        s_cup=s3;
    end
    
    if( (real(s_cup)>0 && imag(s_cup)>0)~=(real(s)>0 && imag(s)>0))
        err=err+1;
    elseif((real(s_cup)<0 && imag(s_cup)<0)~=(real(s)<0 && imag(s)<0) )
        err=err+1;
    elseif((real(s_cup)<0 && imag(s_cup)>0)~=(real(s)<0 && imag(s)>0) )
        err=err+1;
    elseif((real(s_cup)>0 && imag(s_cup)<0)~=(real(s)>0 && imag(s)<0) )
        err=err+1;
    end
    
    
    
       
end

simulated_SER=err/N
Ps=2*qfunc(sqrt(EsNo))-(qfunc(sqrt(EsNo)))^2



\end{lstlisting}
\end{otherlanguage}




\selectlanguage{greek}{
Ο αλγόριθμος πραγματοποιήθηκε $N=10^6$ φορές. Η αλλαγή του \foreignlanguage{english}{EsNo\_dB} γίνεται κάθε
φορά χειροκίνητα για να πάρουμε το κάθε αποτέλεσμα.

Η τιμη \foreignlanguage{english}{simulatedSER} αφορά το πειραματικό \foreignlanguage{english}{SER}, το οποίο είναι ο αριθμός των εσφαλμένων \foreignlanguage{english}{bits}(ο μετρητής), πρός τον αριθμό $Ν$ των επαναλήψεων. \\
Το θεωρητικό \foreignlanguage{english}{SER} δίνεται με την χρήση της συνάρτησης $Q$ απο τον τύπο:
}

\[P_s=2Q\left(\sqrt{\frac{E_s}{N_0}}\right)-Q^2\left(\sqrt{\frac{E_s}{N_0}}\right)\] \\ \\ 

Τα σύμβολα αποτελούν μιγαδικούς αριθμούς, συνεπώς, πραγματοποιήθηκε σύγκριση μεταξύ πραγματικού και φανταστικού μέρους των αριθμών.

\subsection{Αποτελέσματα}

Θέτουμε στην τιμή του \foreignlanguage{english}{EsNo\_dB} το 5 και παίρνουμε:
\foreignlanguage{english}{
\begin{verbatim}
simulatedSER = \\
    0.0743\\
Ps =\\
    0.0739\\
\end{verbatim}}

Για \foreignlanguage{english}{EsNo\_dB}$=8$:
\foreignlanguage{english}{
\begin{verbatim}
simulatedSER = \\
    0.0120\\
Ps =\\
    0.0120\\
\end{verbatim}}

Για \foreignlanguage{english}{EsNo\_dB}$=10$:
\foreignlanguage{english}{
\begin{verbatim}
simulatedSER = \\
    0.0016\\
Ps =\\
    0.0016\\
\end{verbatim}}

\end{document}
